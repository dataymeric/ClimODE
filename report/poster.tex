\documentclass[landscape,a0paper,fontscale=0.292]{baposter}

\usepackage[vlined]{algorithm2e}
\usepackage{times}
\usepackage{calc}
\usepackage{url}
\usepackage{graphicx}
\usepackage{amsmath}
\usepackage{amssymb}
\usepackage{relsize}
\usepackage{multirow}
\usepackage{booktabs}

\usepackage{graphicx}
\usepackage{multicol}
\usepackage[T1]{fontenc}
\usepackage{ae}
\usepackage{enumitem}

\usepackage{colortbl}
\usepackage{xcolor}
%\usepackage{gensymb} % for \degree
\graphicspath{{figs/}}

\setlist[itemize]{leftmargin=*,nosep}
    \setlength{\columnsep}{0.7em}
    \setlength{\columnseprule}{0mm}

\setlist[enumerate]{leftmargin=2.5em,nosep}
    \setlength{\columnsep}{1.0em}
    \setlength{\columnseprule}{0mm}

% %%%%%%%%%%%%%%%%%%%%%%%%%%%%%%%%%%%%%%%%%%%%%%%%%%%%%%%%%%%%%%%%%%%%%%%%%%%%%%%%
% % Save space in lists. Use this after the opening of the list
% %%%%%%%%%%%%%%%%%%%%%%%%%%%%%%%%%%%%%%%%%%%%%%%%%%%%%%%%%%%%%%%%%%%%%%%%%%%%%%%%
% \newcommand{\compresslist}{%
% \setlength{\itemsep}{0pt}%
% \setlength{\itemsep}{0pt}%
% \setlength{\parskip}{0pt}%
% \setlength{\parsep}{0pt}%
% }
\renewcommand{\rmdefault}{ptm} % Arial
\renewcommand{\sfdefault}{ptm} % Arial

\newcommand{\vn}{\boldsymbol{n}}
\newcommand{\vl}{\boldsymbol{l}}
\newcommand{\vM}{\mathbf{M}}
\newcommand{\vN}{\mathbf{N}}
\newcommand{\vL}{\mathbf{L}}
%%%%%%%%%%%%%%%%%%%%%%%%%%%%%%%%%%%%%%%%%%%%%%%%%%%%%%%%%%%%%%%%%%%%%%%%%%%%%
%% Begin of Document
%%%%%%%%%%%%%%%%%%%%%%%%%%%%%%%%%%%%%%%%%%%%%%%%%%%%%%%%%%%%%%%%%%%%%%%%%%%%%
\begin{document}
%%%%%%%%%%%%%%%%%%%%%%%%%%%%%%%%%%%%%%%%%%%%%%%%%%%%%%%%%%%%%%%%%%%%%%%%%%%%%
%% Here starts the poster
%%---------------------------------------------------------------------------
%% Format it to your taste with the options
%%%%%%%%%%%%%%%%%%%%%%%%%%%%%%%%%%%%%%%%%%%%%%%%%%%%%%%%%%%%%%%%%%%%%%%%%%%%%
\begin{poster}{
    % Show grid to help with alignment
    grid=false,
    columns=5,
    % Column spacing
    colspacing=0.7em,
    % Color style
    headerColorOne=cyan!20!white!90!black, % changer les couleurs
    borderColor=cyan!30!white!90!black,    % changer les couleurs
    % Format of textbox
    textborder=faded,
    % Format of text header
    headerborder=open,
    headershape=roundedright,
    headershade=plain,
    background=none,
    bgColorOne=cyan!10!white,
    headerheight=0.12\textheight
}
% Eye Catcher
{
    \includegraphics[width=0.1\linewidth]{logo/Sorbonne}
}
% Title
{
    \sc\huge\bf ClimODE: Climate Forecasting With Physics-informed Neural ODEs
}
% Authors
{
    \vspace{0.3em} Aymeric Delefosse \enspace Mathis Koroglu \enspace Charles Vin %\\[0.2em]
    % préciser qu'on n'est pas les original authors ou pas ?
}
% Other
{
    \begin{tabular}{c}
        \raisebox{-1.0\height}{\includegraphics[width=0.15\linewidth]{logo/ICLR-logo}}\\
        % \raisebox{-0.7\height}{\includegraphics[width=0.16\linewidth]{images/QRCode_Link.pdf}}
    \end{tabular}
}

%%%%%%%%%%%%%%%%%%%%%%%%%%%%%%%%%%%%%%%%%%%%%%%%%%%%%%%%%%%%%%%%%%%%%%%%%%%%%%
%%% Now define the boxes that make up the poster
%%%---------------------------------------------------------------------------
%%% Each box has a name and can be placed absolutely or relatively.
%%% The only inconvenience is that you can only specify a relative position 
%%% towards an already declared box. So if you have a box attached to the 
%%% bottom, one to the top and a third one which should be inbetween, you 
%%% have to specify the top and bottom boxes before you specify the middle 
%%% box.
%%%%%%%%%%%%%%%%%%%%%%%%%%%%%%%%%%%%%%%%%%%%%%%%%%%%%%%%%%%%%%%%%%%%%%%%%%%%%%

%%%%%%%%%%%%%%%%%%%%%%%%%%%%%%%%%%%%%%%%%%%%%%%%%%%%%%%%%%%%%%%%%%%%%%%%%%%%%%
\headerbox{\bf\color{blue} Problem Definition and Contribution}{name=contribution,column=0,row=0,span=2}{
    \textbf{\color{blue}Goal:} Enhancing climate forecasting by integrating physics-informed neural ordinary differential equations (ODEs) with uncertainty quantification. 

    \textbf{\color{blue}Motivations:}
    \begin{itemize}
        \item Existing models neglect the underlying physics, lack of uncertainty quantification and are computationally intensive.
        \item Enhance efficiency and effectiveness in global and regional weather prediction tasks.
    \end{itemize}  

    % \textbf{\color{blue}Contributions:}
    % \begin{itemize}
    %     \item A continuous-time neural advection PDE climate model, and deriving its ODE system tailored to numerical weather prediction.
    %     \item Introducing a flow velocity network that integrates local convolutions, long-range attention in the ambient space, and a Gaussian emission network for predicting uncertainties and source variations.
    %     \item  Demonstrating that all components bring performance improvements, with the advection and emission model having the largest, and attention the least effect.
    % \end{itemize}  
}

\headerbox{\bf\color{blue} Problem Formulation}{name=formulation,column=0,below=contribution,span=2}{
    %\textbf{\color{blue}Assumption:} Orthographic camera with linear radiometric response and directional lighting.
    \textbf{\color{blue}Statistical Mechanics:} Weather can be described as a spatial movement of quantities over time, governed by the partial differential continuity equation:
    \vspace{-0.5em}
    \begin{equation*}
        \underbrace{\frac{du}{dt}}_{\text{time evolution }\dot u} + \underbrace{\overbrace{\mathbf{v} \cdot \nabla u}^{\text{transport}} + \overbrace{u \nabla \cdot \mathbf{v}}^{\text{compression}}}_{\text{advection}} = \underbrace{s}_{\text{sources}},
    \vspace{-0.5em}
    \end{equation*}
    where $u(x, t)$ is a quantity evolving over space $\mathbf{x}$ and time $t$ driven by a flow's velocity $\mathbf{v}(\mathbf{x}, t)$.

    % \begin{center}
    %     \includegraphics[width=\textwidth]{advection.pdf}
    % \end{center}
    % Figure 1 du papier ?

    \textbf{\color{blue}Main Idea:} We solve the continuity equation over entire Earth as a system of neural ODEs. We learn the flow $\mathbf{v}$ as a neural network that uses both global attention and local convolutions and address source variations via a probabilistic emission model that quantifies prediction uncertainties.
}

\headerbox{\bf\color{blue} Method}{name=abstract,column=0,below=formulation,span=2}{
    \textbf{\color{blue}Network Architecture:} 
    \vspace{-0.5em}
    \begin{center}
        \includegraphics[width=\textwidth]{pipeline.pdf}
    \end{center}

    \textbf{\color{blue}Loss Function:} Negative log-likelihood of the observations $\mathbf{y}_i \in \mathbb{R}^{K \times H \times W}$ at times $t_i$ :
    \vspace{-1em}
    \begin{equation*}
        \mathcal{L}_{\theta} = - \frac{1}{NKHW} \sum_{i=1}^{N} \left( \log \mathcal{N}\left(\mathbf{y}_{i}|\mathbf{u}(t_{i}) + \boldsymbol{\mu}(t_{i}), \text{diag}\ \boldsymbol{\sigma}^{2}(t_{i})\right) + \log \mathcal{N}_{+}\left(\boldsymbol{\sigma}(t_{i})|\boldsymbol{0},\lambda_{\sigma}^{2}I\right) \right)
    \end{equation*}
    % \begin{minipage}[t]{0.48\linewidth}
    %     \textbf{\color{blue}Discretization of Lighting Space:}
    %     \vspace{-0.2em}
    %     \begin{center}
    %         \includegraphics[width=\textwidth]{images/lighting_discretization.png}
    %     \end{center}
    %     \vspace{-0.7em}
    %     \begin{itemize}
    %         \item Illustration of the coordinate system (left)
    %         \item Example discretization (right)
    %     \end{itemize}
    % \end{minipage}
    % \hfill
    % \begin{minipage}[t]{0.48\linewidth}
    %     \textbf{\color{blue}Loss Function for Lighting Estimation:}
    %     \vspace{-0.6em}
    %     \begin{equation*}
    %         \mathcal{L}_{\text{Light}} = \lambda_{l_a} \mathcal{L}_{l_a} + \lambda_{l_e} \mathcal{L}_{l_e} + \lambda_e \mathcal{L}_e
    %         \vspace{-0.6em}
    %     \end{equation*}
    %     \begin{itemize}
    %         \item $\mathcal{L}_{l_a}$: azimuth classification loss
    %         \item $\mathcal{L}_{l_e}$: elevation classification loss
    %         \item $\mathcal{L}_e$: light intensity classification loss
    %     \end{itemize}

    %     \vspace{0.5em} 
    %     \textbf{\color{blue}Loss function for Normal Estimation:}
    %     \vspace{-0.6em}
    %     \begin{align*}
    %         \mathcal{L}_{\text{Normal}} = \frac{1}{hw} \sum_{i}^{hw} \left(1 - \vn_i^\top \tilde{\vn}_{i} \right)
    %     \end{align*}
    %     \begin{itemize}
    %         \vspace{-0.6em}
    %         \item Cosine similarity loss
    %     \end{itemize}
    % \end{minipage}
}
%
\headerbox{\bf\color{blue} Overview}{name=contribution,column=2,row=0,span=4}{
    % \begin{minipage}[c]{0.4\textwidth}
    %     \textbf{\color{ctitle}Contributions:}
    %     \vspace{0.2em}
    %     \begin{itemize}
    %         \item Novel problem: 3D neural reflectance field optimization from single-view images captured under different point lights.
    %         \item Exploit monocular shading and shadow cues to jointly recover the geometry and BRDFs of a scene.
    %         \item Adopt an efficient online shadow computation to fully exploit the information-rich shading and shadow cues.
    %         \item Experiments on multiple challenging datasets show that S$^3$-NeRF can faithfully reconstruct a complete scene geometry from single-view images and is robust to depth discontinuity.
    %     \end{itemize}  
    % \end{minipage}\hfill
    % \begin{minipage}[c]{0.6\textwidth}
    %     \begin{center}
    %         \hfill\includegraphics[width=0.98\textwidth]{images/intro.pdf}\\
    %         \vspace{-0.2em}
    %         \makebox[0.4\textwidth]{\footnotesize (a) Different capturing setups}
    %         \makebox[0.59\textwidth]{\footnotesize (b) Comparison of different scene representations}
    %     \end{center}
         
    % \end{minipage}
}
\headerbox{\bf\color{blue} Experiments \& Results}{name=results,column=2,below=contribution,span=4}{
    % \begin{minipage}[t]{0.50\textwidth}
    %     \textbf{\color{blue}Synthetic Training Datasets:} 
    %     \vspace{-0.8em}
    %     \begin{center}
    %         \includegraphics[width=\textwidth]{images/syn_samples.pdf}
    %     \end{center}

    %     \vspace{0.1em}
    %     \textbf{\color{blue}Lighting Distribution of Real Datasets:} 
    %     \vspace{-0.2em}
    %     \begin{center}
    %         \includegraphics[width=0.95\textwidth]{images/real_lighting_dist.pdf}
    %     \end{center}
    % \end{minipage}\hfill
    % \begin{minipage}[t]{0.47\textwidth}
    %     \textbf{\color{blue}Results on DiLiGenT Benchmark~[2]:} 
    %     \vspace{-0.2em}
    %     \begin{center}
    %         \includegraphics[width=0.98\textwidth]{images/quant_diligent.pdf}
    %     \end{center}
    % \end{minipage}

    % \vspace{0.7em}
    % \textbf{\color{blue}Quantitative Results on {\sc Bunny} Rendered with $100$ MERL BRDFs:} \\
    % \vspace{-1.0em}
    % \begin{minipage}[t]{0.23\textwidth}
    %     \vspace{-9.5em}
    %     \begin{center}
    %         \includegraphics[width=0.48\textwidth]{images/MERL_directions.jpg}
    %         \includegraphics[width=0.48\textwidth]{images/bunny_normal.png}\\
    %         \vspace{-0.5em}
    %         \makebox[0.48\textwidth]{\scriptsize (a) Light source} 
    %         \makebox[0.48\textwidth]{\scriptsize (b) {\sc Bunny}} 
    %     \end{center}
    % \end{minipage}
    % \begin{minipage}[t]{0.76\textwidth}
    %     \begin{center}
    %         \includegraphics[width=\textwidth]{images/Bunny_100MERL.pdf}
    %     \end{center}
    % \end{minipage}

    % \textbf{\color{blue}Quantitaive Comparison on {\sc Harvest} in DiLiGenT Benchmark:} \\
    % \begin{center}
    %     \vspace{-0.6cm}
    %     \includegraphics[width=\textwidth]{images/qual_diligent_harvest.pdf}
    % \end{center}

    % \vspace{-0.8em}
    % \begin{minipage}[t]{0.72\textwidth}
    %     \textbf{\color{blue}Qualitative Results on Gourd\&Apple Dataset~[11] and Light Stage Data Gallery~[12]:} \\
    %     \vspace{-1.5em}
    %     \begin{center}
    %         \includegraphics[width=\textwidth]{images/qual_others.pdf}
    %     \end{center}
    % \end{minipage}
    % \begin{minipage}[t]{0.28\textwidth}
    % \textbf{\color{blue}References:} \\
    %     \vspace{-1.0em}
    %     \begin{enumerate}[label={[\arabic*]}]
    %         \footnotesize
    %         \item UPS-FCN [Chen~\emph{et al.}, ECCV8]
    %         \item DiLiGenT [Shi~\emph{et al.}, TPAMI19]
    %         \item AM07 [Alldrin~\emph{et al.}, ICCV07]
    %         \item SM10 [Shi~\emph{et al.}, CVPR10]
    %         \item WT13 [Wu and Tan, CVPR13]
    %         \item LM13 [Lu~\emph{et al.}, CVPR13]
    %         \item PF14 [Papadhimitri and Favaro, IJCV14]
    %         \item LC18 [Lu~\emph{et al.}, TPAMI18]
    %         \item L2 [Woodham, OE1980]
    %         \item IS18 [Ikehata, ECCV18]
    %         \item Gourd\&Apple [Alldrin~\emph{et al.}, CVPR08]
    %         \item Light Stage [Einarsson~\emph{et al.}, EGSR06]
    %     \end{enumerate}
    % \end{minipage}
}
\end{poster}
\end{document}
